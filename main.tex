\documentclass{yLectureNote}
\title{Compsci 796A/B}
\subtitle{Notebook}
\author{Bo Pang}
\date{\today}
\yLanguage{English/Chinese}

\professor{Master of Science}

\usepackage{lipsum}
\usepackage[UTF8]{ctex}

\begin{document}
\titleOne

\yTableOfContent



\chapter{绪论}
\printMarginPartialToc


\section{Mathematical Argumentation}
\classDate{11}{2}{2024}
\nextSerie

\lipsum[1]

\marginTips{Don't remember to practice your math! It is the only way to get through!}


\subsection{Using Conditions}
\lipsum[2]中英混排。还得是刘谦啊,这次所有观众是他的拖,熟悉的声音和动作,五年后再次登上春晚,依旧是让人震惊,高清镜头下的手法,真的太牛了,完全看不出来,但真的好神奇,撕纸牌的那一段,真解 ya,全场沸腾了,跟着他一起做,虽然我猜到蕞后的两个半张牌,一定可以凑成完整的一张,但为什么能对上,我就不知道啦,估计是一道我无法理解的高深数学题。\textbf{中英混排。}
\marginCritical*{I should reread this part, as I didn't understood it well... Maybe it's because it's in Latin?}

\lipsum[3]\marginElement{\marginTitle{Title} \lipsum*[7]}

\checkInfo{Yeaaaah!}{\lipsum[4]\lipsum[2]}


\section{Mathematical Proofs}
\begin{theorem}[Pythagoras]
  Assuming we have a rectangle triangle with the hypotenuse named c. Then:
  \[
    a^2 + b^2 = c^2
  \]
\end{theorem}


\marginInfo*{This formula is really important:
  \begin{equation}
    a^2 + b^2 = c^2
  \end{equation}
}
\nextSerie

\yOrnament
\end{document}
